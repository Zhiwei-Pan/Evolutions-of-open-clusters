
\documentclass[modern]{aastex63}
\usepackage{graphicx}
\usepackage{subfigure}
\usepackage{multirow}
\usepackage{comment}
\usepackage{natbib}
\usepackage{hyperref}
\usepackage{mathtools}
\usepackage{longtable}
\usepackage{mathrsfs}
\usepackage{fontenc}
\usepackage{color}
\usepackage{url}
\usepackage{hyperref}
\usepackage{gensymb}
\usepackage{pifont}



\begin{document}
\title{Final Project Reports}
\author{Yu Qiu}
\affiliation{Kavli Institute for Astronomy and Astrophysic \\
Peking University \\
Beijing 100871, China}
\author{Yuxuan Pang}
\affiliation{Kavli Institute for Astronomy and Astrophysic \\
Peking University \\
Beijing 100871, China}
\author{Zhiwei Pan}
\affiliation{Kavli Institute for Astronomy and Astrophysic \\
Peking University \\
Beijing 100871, China}
\email{1901110222.pku.edu.cn}


\begin{abstract}
This paper is a naive “review” on galaxy formtion by introducing some backgrounds and summarizing Susan Kassin's colloquim. This colloqium is about thier recent study on the formation and evolution of disk and spheroidal galaxies, which will challenge the classic scenarios. They use Hubble images and Keck spectroscopy at $z=0-2.5$ to study the kinematic evolution of disk galaxies and find that disks are gradually settling from disordered motion at high redshift to rotation dominance at low redshift. They also find that the massive galaxies evolve faster and the low-mass galaxy below a typical mass even can't form a disk. As for the spheriodal formation problem, they measure the star formation histories by fitting models and find mainly three scenarios, which indicates that the evolution of high redshift and high mass is faster. At last, they demonstrate that the spectroscopy of JWST in the future will help them a lot, such as creating the kinematic maps of the earliest galaxies.


\end{abstract}

\keywords{disk galaxy --- kinematics --- SFH --- JWST}


\section{Introduction}\label{sec:intro}
Galaxy is a gravitational surported system with gas, stars, Dark matter, which has mainly two types in univserse: disk galaxy and early-type galaxy. Disk galaxy is rotationally supported, which means that stars and gas rotate around the center of disk; While early-type galaxy with many old and low-mass stars is dispersion dominated. 

However, the formation and evolution of galaxy is far from understood. People are always tring to solve out such problems: How and when are disks assembled? How and when do disks obtain their current well-ordered state? How and when do quiescent spheriods form? In history there existed some theoretical models such as the gas-collapsing model to interprete these questions, but in this colloquim, the observation results of this speaker challenge the physical picture of typical model. In this summary, I aim to follow the logic of this colloquim and introduce its three parts.

In Section \ref{disk}, I will introduce their new understanding of disk formation process. In Section \ref{sph}, I will discuss spheriod formation with their fitted star formation history. In Section , I will do the outlook about what JWST can do in the future. In Section \ref{con}, I will present the conculsions of this colloquim. They adopt a $\Lambda$CDM cosmology defined with $(h, \Omega_m,\Omega_{\Lambda})=(0.7, 0.3, 0.7)$.

\bigskip
\bigskip
\section{Disk Formation}\label{disk}
In this part, I will firstly introduce some analytical models and their predictions in Section \ref{model}. Then I will show the observation results and measurements of some galaxy physical properties. At last, I will discuss the critical mass below which the disk will not form in Section \ref{mass} and the characteristics of disks assembly over cosmic time in Section \ref{assemble}

\subsection{Models of Disk Formation and Evolution}\label{model}
We know that the over-density regions of dark matter will cluster and collapse into a galaxy and \citet{1978MNRAS.183..341W} found that baryonic gas dissipation must play a significant role in the formation of disc galaxies. While \citet{1980MNRAS.193..189F} argued that \citet{1978MNRAS.183..341W} didn't discuss the angular momentum problem, so they firstly consider a gas-collapsing model in extended dark matter halo with hierarchical clustering. The key assumption here is that these systems get their angular momentum by tidal torques and the angular momentum is conserved through galaxy collapse, which is very crucial to disk forming. \citet{1997ApJ...482..659D} extended previous models and developed a scenrio that links the surface brightness of the resultant disk to the mass and angular momentum of its protogalaxy: Low surface brightness galaxies form from low-mass and high angular momentum protogalaxies. These models predict that disks start off well-ordered and they grow in radius in an regular manner called "onion skin". 




\subsection{observations}\label{observation}
They did observations of galaxy kinematics at a range of redshifts aimming to examine the picture discribed by models. To study the evolution of galaxy kinematics from z=3 to now, they need spectra and Hubble images for a hundreds of representative galaxies over a significant range in redshift and stellar mass. They firstly selected some star-forming galaxies on the "main sequence" as observation targets with no cuts on morphology, which can reduce selection bias. Then they did spectroscopy and multi-band photometry in DEEP2 \& SIGMA Surveys \citep{2017ApJ...843...46S} and got Hubble images from AEGIS \& CANDELS \citep{2007ApJ...660L...1D,2011ApJS..197...35G}. 
sure $\sigma$ and also the first one to correct both $V_{rot}$ and $\sigma$ for the effects of seeing. They also correct $V_{rot}$ for inclination using Hubble image.


\subsection{mass of disk formation}\label{mass}
By using series of observations introduced above, they studied the TFR at local galaxies 


\subsection{Assembly of disks over cosmic time}\label{assemble}
After concluding that Tully-Fisher relation falls apart for local low mass galaxies and local low-mass star-forming galaxies are often not disks galaxies,

\subsection{comparisions between models and observations}
In this subsection, I will make a little review to compare the difference between predictions from models and conclusion of this colloquim as shown in Table.\ref{table:vs}. 

\begin{deluxetable*}{cc}
\tablenum{1}
\tablecaption{Models v.s. Observations\label{table:vs}}
\tablewidth{0pt}
\tablehead{\colhead{Predictions from models} & \colhead{Conclusions of colloquim} }
\startdata
Disks star off well-ordered and thin & Disks start off with lots of disordered motions\\
Disks grow in radius in an “onion skin” manner & Disks grow in radius, but not in an orderly manner\\
The Tully-Fisher relation has little scatter & Tully-Fisher has large scatter to low rotation velocity\\
They thicken with time & They lose disordered motions and “thin out” with time
\enddata
\end{deluxetable*}



\bigskip
\bigskip
\section{Spheriod Formation}\label{sph}

\subsection{theories of spheriod formation}
After discussing the formation of star-forming disk galaxies, in this section the speaker focus on the problem that how and when galaxies become “red and dead” spheriod galaxies.

The classical picture for red, early-type galaxies is that they form in a single-burst very early in the universe \citep{1975MNRAS.173..671L}. However, \citet{1995IAUS..164..249F} have already successfully challenged this picture, finding that quiescent galaxies increase in numbers by a factor of 2-4 over the last 8 billion years since z~1, so all could not have formed in a single burst in the early universe. \citet{2000AJ....119.1645T} also found that early-type galaxies span a large range of ages and thus showed that blue galaxies can become red via different mechanisms occurring at different times and masses. Besides, there exist some theories that mergers of disks can also form an early type \citep{1977egsp.conf..401T}. 

One of the most important way to study the path from star-forming disks to quiescent spheroids ("quenching”) is to infer galaxy formation histories (SFHs) from their past stellar populations (e.g. \citealt{2015MNRAS.448.3484M}). So the speaker investigate this quenching path problem with measurements of the SFHs of quiescent glaxies. \citep{2016ApJ...832...79P}

\subsection{measurements of SFHs}
They used HST/WFC3-F160W-selected catalogs from CANDELS \citep{2011ApJS..197...36K} and fitted the photometry of 845 quiescent galaxies with a library of 500,000 galaxy SED mo



\section{conclusions}\label{con}
In this colloquim, the speaker mainly introduce three topics.

The first topic is about the disk formation. There is a critical mass of disk formation in the local universe, above which disks always form, but below which they find it hard to. They also found that at high z, galaxies are highly disordered and only just assembling. They later settle to disks. Besides, massive galaxies are always the earliest and fastest evolved.

Queicent spheriod formation is the second topic. She conclude that galaxy quenching time evolves with mass and redshift: it is faster at low mass and high redshift.

At last, She did an outlook about JWST, which will enable them to determine the dynamical state of galaxies soon after reionization and create kinematic maps.


\acknowledgments
We sincerely thank to Xiaoting Fu's generous help and advice. 

\newpage
\bibliography{colloquim}{}
\bibliographystyle{aasjournal1}






\end{document}

% End of file `sample63.tex'.
