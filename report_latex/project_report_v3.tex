
\documentclass[modern]{aastex63}
\usepackage{graphicx}
\usepackage{subfigure}
\usepackage{multirow}
\usepackage{comment}
\usepackage{natbib}
\usepackage{hyperref}
\usepackage{mathtools}
\usepackage{longtable}
\usepackage{mathrsfs}
\usepackage{fontenc}
\usepackage{color}
\usepackage{url}
\usepackage{hyperref}
\usepackage{gensymb}
\usepackage{pifont}



\begin{document}
\title{Final Project Reports}
\author{Yu Qiu}
\affiliation{Kavli Institute for Astronomy and Astrophysic \\
Peking University \\
Beijing 100871, China}
\author{Yuxuan Pang}
\affiliation{Kavli Institute for Astronomy and Astrophysic \\
Peking University \\
Beijing 100871, China}
\author{Zhiwei Pan}
\affiliation{Kavli Institute for Astronomy and Astrophysic \\
Peking University \\
Beijing 100871, China}
\email{1901110222.pku.edu.cn}
\email{1901110223.pku.edu.cn}
\email{1901110222.pku.edu.cn}

\begin{abstract}
In this project, we study the methods of open cluster membership selection and discuss the appliable conditions for simple method. We also use the photomerty and astrometry information from Gaia DR2 to estimate the age of individual cluster from literature by CMD isochrone fitting. At last, we study the evolution peroperties of open cluster.


\end{abstract}

\keywords{open cluster --- astrometry --- CMD --- evolution}


\section{Introduction}\label{sec:intro}
An Open Cluster(OC) is a group of up to a few thousand stars that were formed from the same giant molecular cloud and have roughly the same age. The study of OCs stellar population, formation and evolution has a great contribution to modern astrophysics. From their star formation process, the assembly and evolution of the Milky Way disc can be studied(\cite{2016A&A...591A..37J},\cite{2016A&A...588A.120C}); from their spatial distribution and motion, the gravitational potential and the perturbations of Milky Way disc could be better understand; from their age distribution, the Galactic structure can be traced(\cite{2014MNRAS.444..290B}). OCs are also popular tracers to follow the metallicity gradient of the Milky Way(\cite{2017MNRAS.470.4363C})and its evolution through time(\cite{2016A&A...585A.150N}). 

In addition to useful tracers of the Milky way, OCs are interesting targets in their own right. The most important characteristic of the OCs is that they are loosely bound by mutual gravitational attraction and will gradually lose their stellar content. Generally, their evolution can be split into three phases: (i) the first lasts for $\sim$ 3 Myr, during which the cluster in embedded in its progenitor molecular cloud and stars are still forming; (ii) then the clusters experience the expulsion of the residual star-forming gas; and finally, (iii) the long term evolutionary phase dominated by both internal and external dynamical processes.
	
\emph{Gaia} second data release in 2018(\cite{2018A&A...616A..12G}) has opened a new era in cluster science. Before \emph{Gaia}, \cite{2013A&A...558A..53K} listed about 3000 OCs candidates, then \cite{2018A&A...618A..93C} use the membership assignment code
UPMASK (Unsupervised Photometric Membership Assignment in Stellar Clusters, \cite{2014A&A...561A..57K}) determine 1229 OCs' members. \cite{2019A&A...623A.108B} applied an automated Bayesian tool, BASE-9, to fit stellar isochrones and finally gives 269 OCs' age. Our project aims to reproduce a general process for analyzing star clusters from data, which divide into three parts. 
In the first part, we try to give a method to distinguish a star cluster from the field stars. The previous method mainly including two kinds of method, first is divided stars into different magnitude band then select a typical radius for each band by eye[citation], and second method is using a PCA program to determine the candidates first(\cite{2018A&A...618A..93C}). Both the method also use the restriction of proper-motion and location of the star. 
In the second part,  

Since their population may cover a wide range of age and some physcical properties, we can investigate the relation between the mass, radius or velocity dispersion and age to trace the evolution of these properties and find out how the clusters dissipate and lose stellar mass and maybe some other interesting things.

Our summary is organised as follows: in Section \ref{selection}, we will talk about selection. In Section \ref{age}, we will talk abput age. In Section \ref{evolution}, we will talk about evolution. In Section \ref{con}, we will make a summary.

\bigskip
\bigskip
\section{Cluster Membership selection}\label{selection}






\bigskip
\bigskip
\section{age determination}\label{age}




\bigskip
\bigskip
\section{evolution of open clusters}\label{evolution}




\bigskip
\bigskip
\section{conclusions}\label{con}
In this project, we study the methods of open cluster membership selection and discuss the appliable conditions for simple method. We also use the photomerty and astrometry information from Gaia DR2 to estimate the age of individual cluster from literature by CMD isochrone fitting. At last, we study the evolution peroperties of open cluster.


\acknowledgments
We sincerely thank to Xiaoting Fu's generous help and advice.

\newpage
\bibliography{colloquim}{}
\bibliographystyle{aasjournal1}






\end{document}

% End of file `sample63.tex'.